%%%%%%%%%%%%%%%%%%%%%%%%%%%%%%%%%%%%%%%%%
% Dreuw & Deselaer's Poster
% LaTeX Template
% Version 1.0 (11/04/13)
%
% Created by:
% Philippe Dreuw and Thomas Deselaers
% http://www-i6.informatik.rwth-aachen.de/~dreuw/latexbeamerposter.php
%
% This template has been downloaded from:
% http://www.LaTeXTemplates.com
%
% License:
% CC BY-NC-SA 3.0 (http://creativecommons.org/licenses/by-nc-sa/3.0/)
%
%%%%%%%%%%%%%%%%%%%%%%%%%%%%%%%%%%%%%%%%%

%----------------------------------------------------------------------------------------
% PACKAGES AND OTHER DOCUMENT CONFIGURATIONS
%----------------------------------------------------------------------------------------

\documentclass[final,hyperref={pdfpagelabels=false}]{beamer}

% Use the beamerposter package for laying out the poster with a portrait orientation and an a0 paper size
\usepackage[size=custom, width=76.2, height=101.6, scale=1.4]{beamerposter}

% Choose beamerposter theme
\usetheme{/ORNL}

% English language/hyphenation
\usepackage[english]{babel}

% For including math equations, theorems, symbols, etc
\usepackage{amsmath,amsthm,amssymb,latexsym}

\usepackage{tikz}
\newcommand\PlaceText[3]{%
\begin{tikzpicture}[remember picture,overlay]
\node[outer sep=0pt,inner sep=0pt,anchor=south west]
  at ([xshift=#1,yshift=-#2]current page.north west) {#3};
\end{tikzpicture}%
}
% Code coloring
\usepackage{listings}
\lstdefinelanguage{SWIG}{%
  keywords={module,include,apply}
}
\definecolor{hellgelb} {rgb}{1.0,1.0,0.8}
\definecolor{darkgreen}{rgb}{0.0,0.2,0.13}
\lstset{%
  backgroundcolor=\color{hellgelb!95!MSUgreen},
  % backgroundcolor=\color{white!92!MSUgreen},
  basicstyle=\bf\ttfamily\small,
  breakautoindent=true,
  breaklines=true,
  captionpos=b,
  commentstyle=\color{orange},
  escapechar=|,
  extendedchars=true,
  float=hbp,
  frame=single,
  identifierstyle=\color{black},
  keywordstyle=\color{blue},
  numberstyle=\tiny,
  showspaces=false,
  showstringspaces=false,
  stringstyle=\color{purple},
  tabsize=2,
}
%\usepackage{times}\usefonttheme{professionalfonts}  % Uncomment to use Times as the main font
%\usefonttheme[onlymath]{serif} % Uncomment to use a Serif font within math environments

% Use bold for everything within the math environment
\boldmath

% Top and bottom rules for tables
\usepackage{booktabs}

% Location of the graphics files
\graphicspath{{figs/}}

% A fix for figure numbering
\usecaptiontemplate{\small\structure{\insertcaptionname~\insertcaptionnumber: }\insertcaption}

\usepackage{subcaption}

%----------------------------------------------------------------------------------------
% TITLE SECTION
%----------------------------------------------------------------------------------------

% Poster title
% \title{\huge Parallel Bounding Volume Hierarchy Implementation using MPI+Kokkos}
\title{\huge Parallel Bounding Volume Hierarchy using MPI+Kokkos}

% Author(s)
\author{D.~Arndt, D.~Lebrun-Grandi{\'e}, A.~Prokopenko, B.~Turcksin}

% Institution(s)
\institute{Oak Ridge National Laboratory}

%----------------------------------------------------------------------------------------
% FOOTER TEXT
%----------------------------------------------------------------------------------------

\newcommand{\leftfoot}{}%{http://www.LaTeXTemplates.com} % Left footer text

\newcommand{\rightfoot}{}%{john@smith.com} % Right footer text

%----------------------------------------------------------------------------------------
\begin{document}

% White space under blocks
\addtobeamertemplate{block end}{}{\vspace*{2ex}}

% The whole poster is enclosed in one beamer frame
\begin{frame}[t,fragile]

% The whole poster consists of two major columns, each of which can be
% subdivided further with another \begin{columns} block - the [t] argument
% aligns each column's content to the top
\begin{columns}[t]

%========================================================================================
% Empty spacer column
\begin{column}{.02\textwidth}\end{column}

% The first column
\begin{column}{.450\textwidth}

%----------------------------------------------------------------------------------------
% DESCRIPTION
%----------------------------------------------------------------------------------------
\begin{block}{\centering Motivation}

  Search for close geometric objects is an important algorithm of many
  scientific applications.
  Examples include computing interpolation for overlapping meshes, finding
  mesh intersections, iterative close point algorithm, collision detection, etc.
  This search procedure has to be distributed for large scale simulations, and
  be able to work on both accelerators (such as GPU) and host side, without a
  need to move the application data.

  We developed a portable search capability based on bounding volume
  hierarchy~(BVH) structure. It is based on MPI/Kokkos, and allows for both
  spatial and nearest search algorithms.
\end{block}

%----------------------------------------------------------------------------------------
% BVH CONSTRUCTION
%----------------------------------------------------------------------------------------
\begin{block}{\centering BVH construction}

  Bounding volume hierarchy (BVH) is a tree structure on a set of geometric
  objects, where each object is associated with a conservative bounding box.

  The construction of a BVH tree is done in several stages:
  \begin{enumerate}
    \item User provided geometric objects are wrapped in axis-aligned bounding
      boxes (AABB).
    \item
      For each point, a Morton code is computed.

      \begin{center}
      \includegraphics[scale=1.5]{figures/an18/fig04-z-curve.png}

        {\small (T.~Karras, NVidia)}
      \end{center}
    \item
      Morton codes are sorted to improve the locality of neighboring objects.
    \item
      The binary tree is constructed fully in parallel.

      \begin{center}
      \includegraphics[scale=1.5]{figures/an18/fig03-bvh.png}

        {\small (T.~Karras, NVidia)}
      \end{center}
    \item The bounding boxes of each node are computed.
  \end{enumerate}

  \

  In distributed setting, there are \textbf{two} trees:
  \begin{itemize}
    \item
      The \textbf{lower tree} is a local BVH tree for on-node data.
    \item
      The \textbf{upper tree} is a BVH tree where the geometric objects are the
      roots of local BVH trees. This tree is replicated among nodes.
  \end{itemize}

\end{block}

%----------------------------------------------------------------------------------------
\end{column} % End of the first column

%========================================================================================
% Empty spacer column
\begin{column}{.03\textwidth}\end{column}

% The second column
\begin{column}{.480\textwidth}

%----------------------------------------------------------------------------------------
% BVH Traversal
%----------------------------------------------------------------------------------------
\begin{block}{\centering BVH traversal}
  Two distinct flavors of search:

  \begin{itemize}
    \item \textbf{Spatial search}

      Example: find all the neighbors within a specified radius $r$ of the target point.
      Internal data structure: \textbf{stack}.

    \item \textbf{Nearest search}

      Example: find the predefined $k$ closest neighbors to the target point
      (kNN).
      Internal data structure: \textbf{priority queue}.

  \end{itemize}

  All queries are done in parallel. Each thread gets a target for search.
  Threads traverse the tree independently starting from the root
  node.

  It is critical to minimize both \textbf{code} and \textbf{data divergence}.
  This is done through sorting the queries so that local threads correspond to
  objects that are close to each other.

\end{block}

%----------------------------------------------------------------------------------------
% KOKKOS
%----------------------------------------------------------------------------------------
\begin{block}{\centering Kokkos}
  What is Kokkos?
  \begin{itemize}
    \item Kokkos is a C++11 Library for performance-portable on-node parallelism
    \item Kokkos users achieve portability by programming to Kokkos
      abstractions (e.g., parallel algorithm, data layout, memory location,
      compute resource)
    \item Kokkos library provides back ends for efficient execution of kernels
      on multiple architectures (multicore CPU, Intel Xeon Phi, NVIDIA GPU, etc.)
  \end{itemize}
\end{block}

%----------------------------------------------------------------------------------------
% API
%----------------------------------------------------------------------------------------
\begin{block}{\centering Proposed API}

\begin{lstlisting}[language=C++]
MPI_Comm comm;
using Geometry = ArborX::Point;
Kokkos::View<Geometry*, DeviceType> points;
// ... setup points ...
ArborX::DistributedSearchTree bvh( comm, points );
using Query = ArborX::Intersects<ArborX::Sphere>;
Kokkos::View<Query*, DeviceType> queries;
// ... setup queries ...
auto [offset, indices, ranks] = query( bvh, queries );
\end{lstlisting}

\end{block}

%----------------------------------------------------------------------------------------
% RESULTS
%----------------------------------------------------------------------------------------
\begin{block}{\centering Results}


10M points randomly distributed in a unit cube, 1M queries. The time is in seconds. Speedup in {\bf bold}.
\begin{table}
  \centering
  \begin{tabular}{l@{\hskip 50pt}cr@{\hskip 50pt}cr@{\hskip 50pt}cr}
    \toprule
                             & \multicolumn{2}{c}{build}    & \multicolumn{2}{c}{knn search} & \multicolumn{2}{c}{radius search} \\
    \midrule
    boost rtree              & 3.40   & {\bf 1.04}   & 7.54       & {\bf 0.57}   & 4.60          & {\bf 0.99} \\
    dtk bvh (serial)         & 3.52   & {\bf 1.00}   & 4.32       & {\bf 1.00}   & 4.56          & {\bf 1.00} \\
    dtk bvh (omp 2 threads)  & 2.09   & {\bf 1.68}   & 2.29       & {\bf 1.89}   & 2.45          & {\bf 1.86} \\
    dtk bvh (omp 4 threads)  & 1.06   & {\bf 3.32}   & 1.13       & {\bf 3.82}   & 1.23          & {\bf 3.71} \\
    dtk bvh (omp 8 threads)  & 0.59   & {\bf 5.97}   & 0.63       & {\bf 6.86}   & 0.66          & {\bf 6.91} \\
    dtk bvh (omp 16 threads) & 0.34   & {\bf 10.35}  & 0.36       & {\bf 12.00}  & 0.38          & {\bf 12.00}\\
    dtk bvh (cuda)           & 0.31   & {\bf 11.35}  & 0.40       & {\bf 10.80}  & 0.31          & {\bf 14.70}\\
    \bottomrule
  \end{tabular}
\end{table}
\begin{center}
  \small
  CPU: Intel Xeon E5-2698 \qquad GPU: NVIDIA Tesla K80
\end{center}

% Test: fixed number of queries (100k) with varying problem size. DTK was run with 4 OpenMP threads.
% \begin{table}
  % \centering
  % \begin{tabular}{lccccccccc}
    % \toprule
    % & \multicolumn{3}{c}{build} & \multicolumn{3}{c}{knn} & \multicolumn{3}{c}{radius} \\
             % & dtk    & boost  & nanoflann & dtk   & boost & nanoflann & dtk    & boost  & nanoflann \\
    % \midrule
    % 1k       & .707ms & .344ms & .338ms    & 1.06s & 1.87s & 2.11s     & 31.8ms & 24.0ms & 19.1ms \\
    % 10k      & 2.25ms & 3.61ms & 3.70ms    & 1.48s & 2.42s & 2.99s     & 48.1ms & 39.4ms & 32.4ms \\
    % 100k     & 19.1ms & 41.7m  & 39.5ms    & 1.72s & 2.69s & 3.52s     & 80.8ms & 68.0ms & 62.0ms \\
    % 1M       & 124.ms & 300.ms & 450.ms    & 2.36s & 3.06s & 4.88s     & .176s  & .163s  & .195s \\
    % \bottomrule
  % \end{tabular}
% \end{table}

\end{block}

%----------------------------------------------------------------------------------------
% INFORMATION
%----------------------------------------------------------------------------------------
\begin{block}{\centering More information}
  github.com/ORNL-CEES/DataTransferKit
  devblogs.nvidia.com/thinking-parallel-part-iii-tree-construction-gpu
\end{block}

%----------------------------------------------------------------------------------------
% End of the second column
\end{column}

% Empty spacer column
\begin{column}{.015\textwidth}\end{column}

% End of all the columns in the poster
\end{columns}
%========================================================================================

\vfill

\begin{center}
\footnotesize
This research was supported by the Exascale Computing Project (17-SC-20-SC), a
collaborative effort of the U.S. Department of Energy Office of Science and the
National Nuclear Security Administration.

Research sponsored by the Laboratory
Directed Research and Development Program of Oak Ridge National Laboratory,
managed by UT-Battelle, LLC, for the U.S. Department of Energy.
\end{center}


\begin{center}
  \includegraphics[height=3.5cm]{logos/ORNL_cut_and_alpha.jpg}\hspace{7ex}
  \includegraphics[height=4cm]{logos/ECP_transparent.png}\hspace{5ex}
  \includegraphics[height=3.5cm]{logos/DOE.png}\hspace{7ex}
\end{center}

% End of the enclosing frame
\end{frame}

\end{document}
